%	------------------------------------------------------------------------------
%
%		발원문
%
%		작성 : 
%		2022년
%		08월
%		08일
%		월요일
%		첫 작업
%
%		8월 군법당 일요정기법회 다녀와서 작성의 필요성 느낌
%

%	\documentclass[25pt, a1paper]{tikzposter}
%	\documentclass[25pt, a0paper, landscape]{tikzposter}
%	\documentclass[25pt, a1paper ]{tikzposter}
	\documentclass[	20pt, 
							a1paper, 
							portrait, %
							margin=0mm, %
							innermargin=10mm,  		%
%							blockverticalspace=4mm, %
							colspace=5mm, 
							subcolspace=0mm
							]{tikzposter}


%	\documentclass[25pt, a1paper]{tikzposter}
%	\documentclass[25pt, a1paper]{tikzposter}
%	\documentclass[25pt, a1paper]{tikzposter}

% 	12pt  14pt 17pt  20pt  25pt
%
%	a0 a1 a2
%
%	landscape  portrait
%

	%% Tikzposter is highly customizable: please see
	%% https://bitbucket.org/surmann/tikzposter/downloads/styleguide.pdf

	%	========================================================== 	Package
		\usepackage{kotex}						% 한글 사용


%% Available themes: see also
%% https://bitbucket.org/surmann/tikzposter/downloads/themes.pdf
%	\usetheme{Default}
%	\usetheme{Rays}
%	\usetheme{Basic}
	\usetheme{Simple}
%	\usetheme{Envelope}
%	\usetheme{Wave}
%	\usetheme{Board}
%	\usetheme{Autumn}
%	\usetheme{Desert}

%% Further changes to the title etc is possible
%	\usetitlestyle{Default}			%
%	\usetitlestyle{Basic}				%
%	\usetitlestyle{Empty}				%
%	\usetitlestyle{Filled}				%
%	\usetitlestyle{Envelope}			%
%	\usetitlestyle{Wave}				%
%	\usetitlestyle{verticalShading}	%


%	\usebackgroundstyle{Default}
%	\usebackgroundstyle{Rays}
%	\usebackgroundstyle{VerticalGradation}
%	\usebackgroundstyle{BottomVerticalGradation}
%	\usebackgroundstyle{Empty}

%	\useblockstyle{Default}
%	\useblockstyle{Basic}
%	\useblockstyle{Minimal}		% 이것은 간단함
%	\useblockstyle{Envelope}		% 
%	\useblockstyle{Corner}		% 사각형
%	\useblockstyle{Slide}			%	띠모양  
	\useblockstyle{TornOut}		% 손그림모양


	\usenotestyle{Default}
%	\usenotestyle{Corner}
%	\usenotestyle{VerticalShading}
%	\usenotestyle{Sticky}

%	\usepackage{fontspec}
%	\setmainfont{FreeSerif}
%	\setsansfont{FreeSans}

%	------------------------------------------------------------------------------ 제목

	\title{발원문1 }

	\author{ 2020년 8월 }

	\institute{부처님 앞에 불자로서 크고 바른 다짐을 고합니다}
%	\titlegraphic{\includegraphics[width=7cm]{IMG_1934}}

	%% Optional title graphic
	%\titlegraphic{\includegraphics[width=7cm]{IMG_1934}}
	%% Uncomment to switch off tikzposter footer
	% \tikzposterlatexaffectionproofoff

\begin{document}

	\maketitle

	\begin{columns}

		\column{0.5}

%	------------------------------------------------------------------------------ 조직
			\block{■  발원문  }
			{
					\setlength{\leftmargini}{7em}
					\setlength{\labelsep} {1em}
				\begin{LARGE}

우주에 충만하사 아니 계신곳 없으시고

영겁에 항상하사 아니 계신 때 없으시는

불보살님께 돌아가나이다
\\


부처님이시어,

이제 마음 거두어 합장하오니

자비의 문을 열고 지혜의 단비를 뿌려 

목마른 저희들 가슴에 보리의 푸른 싹을 돋게 하고서

항상 욕심 많고 성 잘내고 어리석어

고통스런 업보의 굴레를 벗지 못하는 저희들은

부처님의 찬란한 해탈세계로 나아기기 원하옵니다


				\end{LARGE}
			}

%	------------------------------------------------------------------------------ 전번
			\block{■  찬불가  }
			{
					\setlength{\leftmargini}{7em}
					\setlength{\labelsep} {1em}
				\begin{LARGE}
					\begin{itemize}
					\item [] 삼귀의
					\item [] 반야심경
					\item [] 보현행원
					\item [] 사홍서원
					\item [] 산회가
					\end{itemize}
				\end{LARGE}
			}


%	------------------------------------------------------------------------------ 회비
			\block{■  회비  }
			{
					\setlength{\leftmargini}{4em}
					\setlength{\labelsep} {1em}
				\begin{LARGE}
					\begin{itemize}
					\item [회비]
					\item [2.] 
					\end{itemize}
				\end{LARGE}
			}



%	------------------------------------------------------------------------------ 수
			\block{■  수 : 7월 21일  제3차 운영위원회 }
			{
					\setlength{\leftmargini}{4em}
					\setlength{\labelsep} {1em}
				\begin{LARGE}
					\begin{itemize}
					\item [일시] 7.21(화) 오후 7시
					\item [장소] 문화관 법당
					\end{itemize}
				\end{LARGE}
			}





	%	====== ====== ====== ====== ====== 
		\column{0.5}


%	------------------------------------------------------------------------------ 국군 법요집
			\block{■  행사 : 국군 법요집 }
			{
					\setlength{\leftmargini}{2em}
					\setlength{\labelsep} {1em}
				\begin{LARGE}
					\begin{itemize}
					\item 1. 법회편
						\begin{itemize}
						\item 우리말 예불문
						\item 한문 예불문
						\item 삼귀의
						\item 보현행원
						\item 우리말 반야심경
						\item 한문 반야심경
						\item 발원문
						\item 청법가
						\item 입정
						\item 설법
						\item 사홍서원
						\item 산회가
						\item 삼귀의 발원문
						\item 법회염송
						\item 수계법회 독송문
						\end{itemize}
					\item 2. 독송편
						\begin{itemize}
						\item 걸림없이 살줄알라
						\item 보왕삼매론
						\item 반석
						\item 마음을 다스리는 글
						\item 행복한 삶
						\item 자비경
						\item 행복경
						\end{itemize}
					\item 3. 상식편
						\begin{itemize}
						\item 간추린 사찰예절
						\item 팔상성도
						\item 불교용어 풀이
						\end{itemize}

					\end{itemize}
				\end{LARGE}
			}



%	------------------------------------------------------------------------------ 행사
			\block{■  행사 : 다라니기도 }
			{
					\setlength{\leftmargini}{4em}
					\setlength{\labelsep} {1em}
				\begin{LARGE}
					\begin{itemize}
					\item [명칭] 다라니기도
					\item [일시] 
					\item [장소] 선문화관
					\end{itemize}
				\end{LARGE}
			}




	\end{columns}




\end{document}


		\begin{huge}
		\end{huge}

		\begin{LARGE}
		\end{LARGE}

		\begin{Large}
		\end{Large}

		\begin{large}
		\end{large}

